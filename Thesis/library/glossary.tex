\newglossaryentry{iot-connection}
{name=IoT Connection, description={ In the proposed system architecture each child layer is “managed” by the parent layer. A connection indicates this management relationship between two devices (e.g cloud-edge) that belong to different layers.}}
\newglossaryentry{iot-management}
{name=IoT Management, description={IoT Management is a generic term that describes various activities which can range from serving as a gateway to performing data analytics on the streamed data while offering additional services such as a GUI dashboard. These activities and services are offered from a parent device to children device.}}

\newglossaryentry{edge-computing}{
name=Edge Computing,
description={Edge computing is a “mesh network of micro data centers that process or store critical data locally and push all received data to a central data center or cloud storage repository.}}

\newglossaryentry{fault-tolerance}{
    name=Fault-Tolerance,
    description={High availability or Fault-Tolerance references to the ability of a system to continue it’s functionality in the case of a critical failure due to the replication of the services to multiple machines in distributed manner.}
}

\newglossaryentry{stakeholder}{
    name=System Stakeholder,
    description={System Stakeholder or simply Stakeholder is the owner of devices that can belong to one or multiple IoT layers, thus forming a vertical slice. Depending on the use case, the stakeholder can also be the system administrator and infrastructure owner.}
}

\newglossaryentry{system-provider}{
name=System Provider,
description={The system provider is the company that produces and sells the system’s software, while also offering core services, such as the marketplace.}
}

\newglossaryentry{south-side}{
    name=South Side,
    description={All IoT objects, within the physical realm, and the edge of the network that communicates directly with those devices, sensors, actuators, and other IoT objects, and collects the data from them, is known collectively as the “South Side.”}
}
\newglossaryentry{north-side}{
    name=North Side,
    description={The Cloud where data is collected, stored, aggregated, analyzed, and turned into information, and the part of the network that communicates with the Cloud, is referred to as the “north side” of the network.}
}

\newglossaryentry{permission}{
    name=Permission,
    description={ Depending on whether the participant nodes need permission from a central authority to participate in the blockchain and write in the ledger, the network is called permission-less or permissioned. }
}
\newglossaryentry{trust}{
    name=Trust (blockchains),
    description={A blockchain is called trustless, if the protocol allows the network to function even when nodes do not trust each other. Otherwise the blokchchain is called trusted}
    }
\newglossaryentry{microdc}{
name=Micro DataCenter,
description={Micro Data centers are data centers with an exceptionally small number of servers racks that is situated geographically near the location of the users that is serving.}}

\newacronym{iot}{IoT}{Internet of Things}

\newacronym{eaas}{EaaS}{Edge as a Service}

\newacronym{ipfs}{IPFS}{Inter Planetary File System}

\newacronym{os}{OS}{Operation System}

\newacronym{mec}{MEC}{Mobile Edge Communications}

\newacronym{vm}{VM}{Virtual Machine}

\newacronym{mqtt}{MQTT}{Message Queuing Telemetry Transport}

\newacronym{rpc}{RPC}{Remote Procedure Call}

\newacronym{rest}{REST}{REpresentational State Transfer}

\newacronym{api}{API}{Application Program Interface}

\newacronym{m2m}{M2M}{Machine-to-Machine}

\newacronym{pow}{PoW}{Proof of Work}

\newacronym{qos}{QoS}{Quality of Service}

\newacronym{qa}{QA}{Quality Assurance}

\newacronym{dmu}{DMU}{Decission Making Unit}

\newacronym{abp}{ABP}{Activation-by-Personalisation}